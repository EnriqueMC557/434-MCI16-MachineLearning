La regresión es un tipo de análisis estadístico que se utiliza para modelar la relación entre variables y predecir valores numéricos en función de variables independientes. Los algoritmos de regresión son técnicas computacionales que se utilizan para encontrar la mejor relación matemática entre los datos disponibles y la variable objetivo.

Algunos de los algoritmos de regresión más comunes son:

\begin{itemize}
	\item \textbf{Regresión lineal}: Es uno de los algoritmos de regresión más simples y ampliamente utilizados. Modela la relación entre dos variables continuas mediante una línea recta que mejor se ajusta a los datos disponibles.
	\item \textbf{Regresión polinomial}: Este algoritmo permite modelar relaciones no lineales entre las variables al usar polinomios de grado mayor que uno para ajustar los datos. Puede capturar patrones más complejos en los datos, pero también puede sufrir de sobreajuste si se utiliza con polinomios de alto grado.
	\item \textbf{Regresión Ridge}: Es un algoritmo de regresión que se utiliza para abordar el problema de multicolinealidad en modelos de regresión lineal. La multicolinealidad ocurre cuando hay alta correlación entre las variables predictoras, lo que puede causar problemas en la estimación de los coeficientes de regresión en un modelo lineal.
	\item \textbf{Regresión de redes neuronales}: Este algoritmo utiliza redes neuronales artificiales para modelar la relación entre variables. Puede capturar patrones complejos en datos grandes y complejos, pero también puede requerir una cantidad significativa de datos para su entrenamiento.
\end{itemize}

Evaluar el desempeño de un modelo de regresión es una parte crítica del proceso de modelado, ya que permite comprender qué tan bien se ajusta el modelo a los datos y cómo se puede mejorar. La evaluación del desempeño de un modelo de regresión implica comparar las predicciones del modelo con los valores reales de la variable objetivo para determinar qué tan preciso es el modelo en sus predicciones. Esto ayuda a determinar la calidad del modelo y su capacidad para generalizar a nuevos datos.

La evaluación del desempeño de un modelo de regresión se puede realizar utilizando varias técnicas y métricas, tales como:

\begin{itemize}
	\item Error cuadrático medio (MSE).
	\item Raíz del error cuadrático medio (RMSE).
	\item Error absoluto medio (MAE).
	\item Error cuadrático relativo (RSE).
	\item Coeficiente de correlación de Pearson (PCC).
	\item Coeficiente de determinación ($R^2$).
\end{itemize}

La evaluación del desempeño de un modelo de regresión es una etapa esencial del proceso de modelado, ya que permite comprender la calidad y la capacidad de generalización del modelo, identificar posibles mejoras y tomar decisiones basadas en los resultados del modelo.
