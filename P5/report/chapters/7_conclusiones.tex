Basado en el desempeño obtenido, se puede concluir que la técnica de selección de atributos utilizando el coeficiente de correlación de Pearson ha demostrado un rendimiento notablemente superior en comparación con las otras dos técnicas, PCA y la técnica experimental.

La técnica de Pearson logró una alta exactitud del 99.58\%, lo que indica que el modelo clasificador pudo predecir correctamente la clase de la mayoría de las instancias del conjunto de datos. Además, alcanzó una sensibilidad del 100\%, lo que significa que pudo identificar correctamente todos los casos positivos de stroke en el conjunto de datos. La precisión también fue alta, con un valor del 99.18\%, lo que implica que la mayoría de las instancias clasificadas como casos positivos fueron realmente positivos. El valor F1 de 99.59\% indica un buen equilibrio entre la precisión y la sensibilidad.

En contraste, las técnicas de PCA y la técnica experimental obtuvieron resultados inferiores. La técnica de PCA logró una exactitud del 61.60\%, lo que sugiere que su capacidad para clasificar las instancias correctamente fue significativamente menor. También tuvo una sensibilidad y precisión más bajas, lo que indica que el modelo fue menos efectivo para identificar los casos positivos de stroke y para clasificar correctamente las instancias. El valor F1 de 62.81\% también muestra un rendimiento moderado en términos de equilibrio entre precisión y sensibilidad.

Estos resultados sugieren que la técnica de selección de atributos basada en el coeficiente de correlación de Pearson puede ser una herramienta valiosa para la detección de stroke. La alta exactitud, sensibilidad y precisión alcanzadas por esta técnica indican que tiene el potencial de identificar de manera efectiva los casos de stroke y minimizar los falsos positivos y falsos negativos. Sin embargo, se necesita más análisis y validación con conjuntos de datos adicionales para confirmar la utilidad y generalidad de estos resultados.

A medida que la investigación en inteligencia artificial y aprendizaje automático avanza, es posible que se logren avances aún mayores en la detección temprana y precisa de stroke. Esto podría conducir a mejoras continuas en la atención médica de emergencia y a una mejora significativa en la calidad de vida de los pacientes que han sufrido un stroke.
