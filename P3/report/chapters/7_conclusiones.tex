Relacionado a los algoritmos implementados para normalización de datos. Se pudo observar que los 3 algoritmos implementados lograron conservar la distribución de los datos originales, lo cual indica que se han implementado de forma exitosa. Hablando en específico de los datos normalizados mediante z-score, cabe destacar que al aplicarlo a atributos continuos cuyo histograma se asemejaba a una distribución normal, se logró enfatizar este comportamiento, lo cual se espera sea de ayuda cuando se desee trabajar con este conjunto de datos en una red neuronal artificial.

Relacionado a los algoritmos de balance de clases con datos sintéticos. En primer lugar, para ambos algoritmos se logró el objetivo de crear más instancias de la clase con menor, logrando pasar de una relación 85-15 a una relación 51-49, lo cuál es una característica ideal de un conjunto de datos. Al revisar las similitudes entre los valores de desviación estándar de cada atributo en cada conjunto de datos generado, fue posible observar que sus valores eran muy similares, lo cuál nos da buenos indicios de que el proceso de balance de clases se realizó correctamente. Por último, al comparar las gráficas de distribución de los conjuntos de datos generados con el conjunto original se logra observar que presentan diferencias considerables, tanto en forma como en magnitud, lo cuál es un indicio de que podría haberse realizado mal el proceso de balanceo de clases. Cabe destacar que para el proceso de balance de clases solamente se realizó la síntesis de datos para la clase menor, generando un total de 200 instancias completamente nuevas.

En conclusión, las técnicas de normalización de datos son de gran importancia ya que permite adecuar nuestros datos a los requerimientos establecidos por nuestros algoritmos de interés. En esta ocasión se implementaron métodos de normalización que conservan la forma de la distribución original, sin embargo, existen algoritmos que buscan lo contrario, como lo es generar una distribución de tipo uniforme. Es parte esencial del especialista de datos el elegir el método apropiado para su caso de uso. Hablando de los algoritmos de balance de clases mediante datos sintéticos, se logró el objetivo principal del balancear, sin embargo, debido a los resultados obtenidos al comparar las distribuciones, se llegan a presentar duda sobre si el proceso fue realizado correctamente, habrá que investigar más a fondo sobre las técnicas y sobre sus métodos de validación.


