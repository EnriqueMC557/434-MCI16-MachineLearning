Es de conocimiento general que todo modelo de inteligencia artificial es tan bueno como los datos con los que fue entrenado.

Dentro de los problemas más comunes que pueden presentarse en los datos se destacan: datos faltantes, problemas de cardinalidad irregular y valores atípicos \cite{Aceves2021}.

Hablando específicamente del problema de datos faltantes, podemos encontrarnos con diversas causas de dicho problema, desde errores en el sistema de adquisición de datos hasta errores humanos \cite{Aceves2021}.

Para solucionar el problema de datos faltantes existen diversos métodos, desde el proceso de eliminar las instancias que cuenten con datos faltantes o bien eliminar los atributos, sin embargo, cuando la cantidad de datos faltantes representa menos del 30\% del total de instancias del atributo, suele ser prudente utilizar algún método de imputación de datos \cite{Aceves2021}.
