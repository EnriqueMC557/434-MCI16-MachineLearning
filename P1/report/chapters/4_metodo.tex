Para el desarrollo de la práctica se utilizó el lenguaje de programación Python en su versión 3.10, el cual fue utilizado para desarrollar una Jupyter Notebook con los requerimientos de la práctica.

El conjunto de datos a utilizar proviene de un repositorio de acceso público de la Secretaría de Salud del Gobierno de México, y consiste en datos tabulares anonimizados con información relevante a casos de COVID-19 en México.

Para la obtención de los datos estadísticos se utilizaron métodos de los paquetes \emph{Numpy} y \emph{Statistics} de Python, los cuales se consumieron dentro de una clase diseñada para agilizar el proceso de obtención de datos estadisticos.

Para la visualización de los datos se utilizaron dos paquetes de Python especializados en la visualización de datos interactiva, \emph{Bokeh} y \emph{Pygal}, los cuales pueden ser instalados en el interprete de Python mediante el gestor de paquetes PIP con los comandos:

\begin{lstlisting}
	pip install Pygal
	pip install bokeh
\end{lstlisting}

Para el caso especial del paquete \emph{Bokeh} se requiere de un paquete adicional, el cual se puede instalar de igual forma con el gestor de paquetes PIP:

\begin{lstlisting}
	pip install selenium
\end{lstlisting}
