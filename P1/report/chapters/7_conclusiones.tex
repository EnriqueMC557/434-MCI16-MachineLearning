Es conocido que un modelo de inteligencia artificial es tan bueno como los datos con los que se haya entrenado, y para estar seguros de la calidad de los datos es necesario conocer su composición y medidas de estadistica descriptiva.

En esta práctica se logró obtener un panorama general del conjunto de datos utilizado mediante un análisis estadistico por atributo, obteniendo así valores máximos, mínimos, promedios, modas y medianas, desviaciones estándar y datos faltantes, entre otros.

Además, mediante la implementación de diversos gráficos fue posible obtener una representación visual de las relaciones existentes entre algunos atributos, así como la distribución de cada uno de los mismos.
